\documentclass[10pt,a4paper,landscape]{article}
\usepackage[utf8]{inputenc}
\usepackage[french]{babel}
\usepackage[T1]{fontenc}
\usepackage{amsmath}
\usepackage{amsfonts}
\usepackage{amssymb}
\usepackage{graphicx}
\usepackage{lscape}
\usepackage{multirow}
\usepackage{array}



\begin{document}

\begin{center}
  \huge Relevé de notes détaillé \VAR{semestre.libelle}
\end{center}

\hspace{-9cm}
\begin{tabular}{|c|c|c|c|c|c|c|}

% Liste des UEs



\hline 
N° & \large\textbf{Prénom} & \large\textbf{Nom} &
% Base de données
\rotatebox{90}{\parbox[c]{2.5cm}{\centering UML }}  &
\rotatebox{90}{\parbox[c]{2.5cm}{\centering BPMN }} &
\rotatebox{90}{\parbox[c]{2.5cm}{\centering Base de données }} &
\rotatebox{90}{\parbox[c]{2.5cm}{\centering UE }}  \\
% Base de données fin




\hline 
\textbf{1} & Marco & Jhalahl & 10 & 10 & 10 & 10 \\ 
 
\hline 

\end{tabular} 


\newpage


\begin{tabular}{|c|>{\centering\arraybackslash}p{3cm}|>{\centering\arraybackslash}p{3cm}|}
  \cline{2-3}
  \multicolumn{1}{c|}{} & \rotatebox{90}{\parbox[c]{2.5cm}{\centering Colonne des Données semis structurées}} & \rotatebox{90}{\parbox[c]{2.5cm}{\centering Colonne2 et Colonne3}} \\
  
  \hline
  Document de synthèse du semestre 1 & Document de synthèse du semestre 2 & Document de synthèse du semestre 3 \\
  \cline{2-3}
  \hline
  Document de synthèse du semestre 4 & Document de synthèse du semestre 5 & Document de synthèse du semestre 6 \\
  \cline{2-3}
  \hline
\end{tabular}














\end{document}